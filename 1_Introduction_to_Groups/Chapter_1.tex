\documentclass[11pt,largemargins]{homework}

\newcommand{\hwname}{------ ------}
\newcommand{\hwemail}{-}
\newcommand{\hwtype}{Chapter}
\newcommand{\hwnum}{1}
\newcommand{\hwclass}{Abstract Algebra}
\newcommand{\hwlecture}{0}
\newcommand{\hwsection}{Z}

\begin{document}
\maketitle


\textbf{\large{End of Chapter Exercises}}

%%%% ONE %%%%
\question
With words, describe each symmetry in $D_3$ (the set of symmetries of an equilateral triangle).

Let $a,b,c$ be the points that form an equilateral triangle.

\quad $R_0$: The identity element of the group; does nothing.

\quad $R_{120}$: A clockwise rotation of $120^o$ about the center.

\quad $R_{240}$: A clockwise rotation of $240^o$ about the center.

\quad $F_a$: A reflection of the triangle keeping point $a$ in place.

\quad $F_b$: A reflection of the triangle keeping point $b$ in place.

\quad $F_c$: A reflection of the triangle keeping point $c$ in place.

%%%% TWO %%%%
\question 
Write out a complete Cayley table for $D_3$. 

\centering
\begin{tabular}{cccclll}
   $D_3$       &  $R_0$     &  $R_{120}$  &  $R_{240}$  & $F_a$     & $F_b$     &  $F_c$  \\
   $R_0$       &  $R_0$     &  $R_{120}$  &  $R_{240}$  & $F_a$     & $F_b$     &  $F_c$  \\
   $R_{120}$   &  $R_{120}$ &  $R_{240}$  &  $R_0$      & $F_c$     & $F_a$     &  $F_b$  \\
   $R_{240}$   &  $R_{240}$ &  $R_0$      &  $R_{120}$  & $F_b$     & $F_c$     &  $F_a$  \\
   $F_a$       &  $F_a$     &  $F_b$      &  $F_c$      & $R_0$     & $R_{120}$ &  $R_{240}$  \\
   $F_b$       &  $F_b$     &  $F_c$      &  $F_a$      & $R_{240}$ & $R_0$     &  $R_{120}$  \\
   $F_c$       &  $F_c$     &  $F_a$      &  $F_b$      & $R_{120}$ & $R_{240}$ &  $R_0$  
\end{tabular}

\raggedright

%%%% THREE %%%%
\question 
Is $D_3$ Abelian?

\quad No, $D_3$ is not Abelian (commutative).

\quad An example of operations on $D_3$ which do not commute is $F_a$ and $F_b$


%%%% FOUR %%%%
\question 
Describe in words the elements of $D_5$ (symmetries of a regular pentagon).


%%%% FIVE %%%%
\question 
For $n\geq 3$, describe the elements of $D_n$. \textit{(Hint: Consider two cases - n even and n odd.)} 
How many elements does $D_n$ have?

\quad \textit{Note: differences are in the reflection of the n-gon. For n odd: each corner is assigned a reflection. For n even: 
half of the corners are assigned a reflection, and half of the sides are assigned a reflection.}

\quad $|D_n| = 2n$

%%%% SIX %%%%
\question 
In $D_n$, explain geometrically why a reflection followed by a reflection must be a rotation.

\quad For the following questions concerning geometrical explanations of composing rotations and reflections, we 
interpret the object being manipulated as a 2-dimensional n-gon. In this fashion, we can assign a "front-side" and "back-side" 
to the object. 

\quad By performing two subsequent reflections. The resulting operation conserves the polarity (sidedness) of the object. Through
this perspective, we can see that the resulting operation is not a reflection, and thus is a rotation (or the identity element).


%%%% SEVEN %%%%
\question 
In $D_n$, explain geometrically why a rotation followed by a rotation must be a rotation.

\quad See explanation for Question 6.

%%%% EIGHT %%%%
\question 
In $D_n$, explain geometrically why a rotation and a reflection taken together in either order must be a reflection.

\quad See explanation for Question 6.

%%%% NINE %%%%
\question 
Associate the number $+1$ with a rotation and the number $-1$ with a reflection. Describe an analogy between 
multiplying these two numbers and multiplying elements of $D_n$.

\quad This is similar to the "sidedness" argument of Question 6 in which subsequent rotation operations are analagous 
to repetitive multiplications of $-1$. The resulting product of the associated numbers ($+1, -1$) is a description of 
which side of the n-gon is facing the reader. 

%%%% TEN %%%%
\question 
If $r_1,r_2,$ and $r_3$ represent rotations from $D_n$ and $f_1,f_2,$ and $f_3$ represent reflections from $D_n$, 
determine whether $r_1r_2f_1r_3f_2f_3r_3$ is a rotation or a reflection.

\quad Using the analogy from Question 9,
\begin{align*}
    r_1r_2f_1r_3f_2f_3r_3 &\simeq (+1)(+1)(-1)(+1)(-1)(-1)(+1)\\
                          &\simeq (-1)\\
                          &\simeq \text{Reflection}
\end{align*}

%%%% ELEVEN %%%%
\question 
Find elements $A$, $B$, and $C$ in $D_4$ such that $AB=BC$ but $A\neq C$. (Thus, "cross cancellation" is not valid.)

\quad Using the notation from this chapter:
$$HR_{90} = D' = R_{90}V$$
$$H \neq V$$

\quad $H$: Horizontal reflection 

\quad $V$: Vertical reflection 

\quad $R_{90}$: $90^o$ rotation about the center

\quad $D'$: Diagonal reflection (SW\&NE corners fixed)

%%%% TWELVE %%%%
\question 
Explain what the following diagram proves about the group $D_n$.

\quad In general, $D_n$ is not Abelian (commutative) because: $$FR_{360/n}\neq R_{360/n}F$$

%%%% THIRTEEN %%%%
\question 
Describe the symmetries of a nonsquare rectangle. Construct the corresponding Cayley table.

\quad The key difference between the symmetries of a nonsquare rectangle and $D_4$ is the absence of $R_{90}$, $R_{270}$ and the 
two diagonal reflections $D$ and $D'$.

\centering
\begin{tabular}{cccclll}
   x         &  $R_0$     &  $R_{180}$  &  $H$       & $V$        \\
   $R_0$     &  $R_0$     &  $R_{180}$  &  $H$       & $V$        \\
   $R_{180}$ &  $R_{180}$ &  $R_0$      &  $V$       & $H$        \\
   $H$       &  $H$       &  $V$        &  $R_0$     & $R_{180}$  \\
   $V$       &  $V$       &  $H$        &  $R_{180}$ & $R_0$  
\end{tabular}

\raggedright

%%%% FOURTEEN %%%%
\question 
Describe the symmetries of a parallelogram that is neither a rectangle nor a rhombus. Describe the symmetries of 
a rhombus that is not a rectangle.

\quad The only symmetry of a nonrectangular-nonrhomboid parallelogram is $R_0 = R_{360}$. 

\quad By assigning each corner of the rhombus a corresponding edge of a nonsquare rectangle we can draw an analogy (isomorphism) 
between the symmetries of a rhombus and a nonsquare rectangle.

%%%% FIFTEEN %%%%
\question 
Describe the symmetries of a noncircular ellipse. Do the same for a hyperbola.

\quad These are the same as the symmetries of the nonsquare rectangle.

%%%% SIXTEEN %%%%
\question 
Consider an infinitely long strip of equally spaced H's:
$$...\text{ H H H H }...$$
Describe the symmetries of this strip. Is the group of symmetries of the strip Abelian?

\quad This is a cross of the group of integers under addition and the group of symmetries of a nonsquare rectangle.

\quad \textit{Claim:} If $G_1$ and $G_2$ are Abelian, then $G_1 \times G_2$ is also Abelian.


%%%% SEVENTEEN %%%%
\question 
For each of the snowflakes in the figure, find the symmetry group and locate the axes of reflective symmetry.


%%%% EIGHTTEEN %%%%
\question 
Determine the symmetry group of the outer shell of the cross section of the human immunodeficiency virus (HIV).


%%%% NINETEEN %%%%
\question 
Does an airplane propeller have a cyclic symmetry group or a dihedral symmetry group?


%%%% TWENTY %%%%
\question 
Bottle caps that are pried off typically have 22 ridges around the rim. Fine the symmetry group of such a cap.


%%%% TWENTY ONE %%%%
\question 
What group theoretic property do upper-case letters F, G, J, K, L, P, Q, R have that is not shared by the 
remaining upper-case letters in the alphabet?


%%%% TWENTY TWO %%%%
\question 
For each design below, determine the symmetry group.


%%%% TWENTY THREE %%%%
\question 
What would the effect be if a six-bladed ceiling fan were designed so that the centerlines of two of the blades were at 
$70^o$ angle and all the other blades were set at $58^o$ angle?





\end{document}




