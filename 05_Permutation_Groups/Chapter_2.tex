\documentclass[11pt,largemargins]{homework}

\newcommand{\hwname}{------ ------}
\newcommand{\hwemail}{-}
\newcommand{\hwtype}{Chapter}
\newcommand{\hwnum}{2}
\newcommand{\hwclass}{Abstract Algebra}
\newcommand{\hwlecture}{0}
\newcommand{\hwsection}{Z}

\begin{document}
\maketitle

\textbf{\large{Important Statements:}}

\underline{Groups:}

\quad 1. \textit{Associativity:} $(ab)c=a(bc) \;\forall\;a,b,c \in G$

\quad 2. \textit{Identity:} 

\quad 3. \textit{Inverses:}

\underline{Uniqueness of the Identity:}

\quad In a group $G$, there is only one identity element.

\underline{Cancellation:}

\quad In a group $G$, the right and left cancellation laws hold; that is, 

\quad $ba=ca \Rightarrow b=c$ and $ab=ac \Rightarrow b=c$

\underline{Uniqueness of Inverses:}

\quad For each element $a$ in a group $G$, there is a unique element $b$ in $G$ such that $ab=ba=e$.

\underline{Socks-Shoes Principle:}

\quad For group elements $a$ and $b$, $(ab)^{-1}=b^{-1}a^{-1}$.

\hfill

\textbf{\large{End of Chapter Exercises}}

%%%% ONE %%%%
\question
Give two reasons why the set of odd integers under addition is not a group.

%%%% TWO %%%%
\question
Referring to Example 13, verify the assertion that subtraction is not associative.


%%%% THREE %%%%
\question
Show that $\{1, 2, 3\}$ under multiplication modulo 4 is not a group but that $\{1,2,3,4\}$ under multiplication 
modulo 5 is a group.

%%%% FOUR %%%%
\question
Show that the group $GL(2, \mathbb{R})$ of Example 9 is non-Abelian by exhibiting a pair of matrices $A$ and $B$ 
in $GL(2, \mathbb{R})$ such that $AB \neq BA$.


%%%% FIVE %%%%
\question
Find the inverse of the element $a$ in $GL(2, \mathbb{Z}_{11})$.


%%%% SIX %%%%
\question
Give an example of group elements $a$ and $b$ with the property that $a^{-1}ba\neq b$.

%%%% SEVEN %%%%
\question
Translate each of the following multiplicative expressions into its additive counterpart. Assume that the operation 
is commutative.

\begin{alphaparts}
    \questionpart
    $a^2b^3$

    \questionpart
    $a^{-2}(b^{-1}c)^2$

    \questionpart
    $(ab^2)^{-3}c^2=e$

\end{alphaparts}



\end{document}








