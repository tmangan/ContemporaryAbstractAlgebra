\documentclass[11pt,largemargins]{homework}

\newcommand{\hwname}{------ ------}
\newcommand{\hwemail}{-}
\newcommand{\hwtype}{Chapter}
\newcommand{\hwnum}{2}
\newcommand{\hwclass}{Abstract Algebra}
\newcommand{\hwlecture}{0}
\newcommand{\hwsection}{Z}

\begin{document}
\maketitle

\textbf{\large{Important Statements:}}

\underline{Groups:}

\quad 1. \textit{Associativity:} $(ab)c=a(bc) \;\forall\;a,b,c \in G$

\quad 2. \textit{Identity:} 

\quad 3. \textit{Inverses:}

\underline{Uniqueness of the Identity:}

\quad In a group $G$, there is only one identity element.

\underline{Cancellation:}

\quad In a group $G$, the right and left cancellation laws hold; that is, 

\quad $ba=ca \Rightarrow b=c$ and $ab=ac \Rightarrow b=c$

\underline{Uniqueness of Inverses:}

\quad For each element $a$ in a group $G$, there is a unique element $b$ in $G$ such that $ab=ba=e$.

\underline{Socks-Shoes Principle:}

\quad For group elements $a$ and $b$, $(ab)^{-1}=b^{-1}a^{-1}$.

\hfill

\textbf{\large{End of Chapter Exercises}}

%%%% ONE %%%%
\question
Give two reasons why the set of odd integers under addition is not a group.

%%%% TWO %%%%
\question
Referring to Example 13, verify the assertion that subtraction is not associative.


%%%% THREE %%%%
\question
Show that $\{1, 2, 3\}$ under multiplication modulo 4 is not a group but that $\{1,2,3,4\}$ under multiplication 
modulo 5 is a group.

%%%% FOUR %%%%
\question
Show that the group $GL(2, \mathbb{R})$ of Example 9 is non-Abelian by exhibiting a pair of matrices $A$ and $B$ 
in $GL(2, \mathbb{R})$ such that $AB \neq BA$.


%%%% FIVE %%%%
\question
Find the inverse of the element $a$ in $GL(2, \mathbb{Z}_{11})$.


%%%% SIX %%%%
\question
Give an example of group elements $a$ and $b$ with the property that $a^{-1}ba\neq b$.

%%%% SEVEN %%%%
\question
Translate each of the following multiplicative expressions into its additive counterpart. Assume that the operation 
is commutative.

\begin{alphaparts}
    \questionpart
    $a^2b^3$

    \questionpart
    $a^{-2}(b^{-1}c)^2$

    \questionpart
    $(ab^2)^{-3}c^2=e$

\end{alphaparts}

%%%% EIGHT %%%%
\question
Show that the set $\{5, 15, 25, 35\}$ is a group under multiplication modulo 40. What is the identity element of this group?
Can you see any relationship between this group and $U(8)$?

%%%% NINE %%%%
\question
Not Available

%%%% TEN %%%%
\question
List the members of $H=\{x^2\;|\; x \in D_4 \}$ and $K=\{x\in D_4 \;|\; x^2=e \}$.

%%%% ELEVEN %%%%
\question
Prove that the set of all $2\times2$ matrices with entries from $\mathbb{R}$ and determinant $+1$ is a group under matrix
multiplication.

%%%% TWELVE %%%%
\question
For any integer $n>2$, show that there are at least two elements in $U(n)$ that satisfy $x^2=1$.

%%%% THIRTEEN %%%%
\question
An abstract algebra teacher intended to give a typist a list of nine integers that form a group under multiplication 
modulo 91. Instead, one of the nine integers was inadvertently left out, so that the list appeares as 
$1, 9, 16, 22, 53, 74, 79, 81$. Which integer was left out? (This really happened!)

%%%% FOURTEEN %%%%
\question
Let $G$ be a group with the following property: Whenever $a,b$, and $c$ belong to $G$ and $ab=ca$, then $b=c$. 
Prove that $G$ is Abelian. ("Cross cancellation" implies commutativity.)

%%%% FIFTEEN %%%%
\question
(Law of Exponents for Abelian Groups) Let $a$ and $b$ be elements of an Abelian group and let $n$ be any integer. Show 
that $(ab)^n=a^nb^n$. Is this also true for non-Abelian groups?

%%%% SIXTEEN %%%%
\question
(Socks-Shoes Property) Draw an analogy between the statement $(ab)^{-1}=b^{-1}a^{-1}$ and the act of putting on and taking off 
your socks and shoes. Find an example that shows that in a group, it is possible to have $(ab)^{-2}\neq b^{-2}a^{-2}$. 
Find distinct nonidentity elements $a$ and $b$ from a non-Abelian group such that $(ab)^{-1}=a^{-1}b^{-1}$.

%%%% SEVENTEEN %%%%
\question
Prove that a group $G$ is Abelian if and only if $(ab)^{-1}=a^{-1}b^{-1}$ for all $a$ and $b$ in $G$.

%%%% EIGHTEEN %%%%
\question
Prove that in a group, $(a^{-1})^{-1} = a$ for all $a$.

%%%% NINETEEN %%%%
\question
For any elements $a$ and $b$ from a group and any integer $n$, prove that $(a^{-1}ba)^n=a^{-1}b^na$.

%%%% TWENTY $$$$
\question 
If $a_1,a_2,...,a_n$ belong to a group, what is the inverse of $a_1a_2...a_n$?

%%%% TWENTY ONE $$$$
\question 
The integers 5 and 15 are among a collection of 12 integers that form a group under mutiplication modulo 56. 
List all 12.

%%%% TWENTY TWO $$$$
\question 
Give an example of a group with 105 elements. Give two examples of groups with 44 elements.

%%%% TWENTY THREE $$$$
\question 
Prove that every group table is a \textit{Latin Square}; that is, each elemtn of the group appears exactly once in each row and 
each column. 

%%%% TWENTY FOUR $$$$
\question 
Construct a Cayley table for $U(12)$.

%%%% TWENTY FIVE $$$$
\question 
Suppose the table below is a group table. Fill in the blank entries.

\centering
\begin{tabular}{llllll}
 x  & $e$ & $a$ & $b$ & $c$ & $d$ \\
$e$ & $e$ & $-$ & $-$ & $-$ & $-$ \\
$a$ & $-$ & $b$ & $-$ & $-$ & $e$ \\
$b$ & $-$ & $c$ & $d$ & $e$ & $-$ \\
$c$ & $-$ & $d$ & $-$ & $a$ & $b$ \\
$d$ & $-$ & $-$ & $-$ & $-$ & $-$
\end{tabular}

\raggedright

%%%% TWENTY SIX $$$$
\question 
Prove that if $(ab)^2=a^2b^2$ in a group $G$, then $ab=ba$.

%%%% TWENTY SEVEN $$$$
\question 
Let $a,b,$ and $c$ be elements of a group. Solve the equation $axb=c$ for $x$. Solve $a^{-1}xa=c$ for $x$.

%%%% TWENTY EIGHT $$$$
\question 
Prove that the set of all rational numbrs of the form $3^m6^n$, where $m$ and $n$ are integers, is a group under mutiplication.

%%%% TWENTY NINE $$$$
\question 
Let $G$ be a finite group. Show that the number of elements $x$ of $G$ such that $x^3=e$ is odd. Show that the number of 
elements $x$ of $G$ such that $x^2\neq e$ is even.

%%%% THIRTY $$$$
\question 
Give an example of a group with elements $a,b,c,d,$ and $x$ such that $axb=cxd$ but $ab\neq cd$. (Hence "middle cancellation" is 
not valid in groups.)

%%%% THIRTY ONE $$$$
\question 
Let $R$ be any rotation in some dihedral group and $F$ any reflection in the same group. Prove that $RFR=F$.

%%%% THIRTY TWO $$$$
\question 
Let $R$ be any rotation in some dihedral group and $F$ any reflection in the same group. Prove that $FRF=R^{-1}$ 
for all integers $k$.

\quad I think there may be a typo in this question. (?)

%%%% THIRTY THREE $$$$
\question 
Suppose that $G$ is a group with the property that for every choice of elements in $G$, $axb=cxd$ implies $ab=cd$. Prove 
that $G$ is Abelian. ("Middle cancellation" implies commutativity.)

%%%% THIRTY FOUR $$$$
\question 
In the dihedral group $D_n$, let $R=R_{360/n}$ and let $F$ be any reflection. Write each of the following products in the 
form $R^i$ or $R^iF$, where $0\leq i < n$. 

\begin{alphaparts}
    \questionpart
    In $D_4$, $FR^{-2}FR^5$
    \questionpart
    In $D_5$, $R^{-3}FR^4FR^{-2}$
    \questionpart
    In $D_6$, $FR^5FR^{-2}F$
\end{alphaparts}
    

%%%% THIRTY FIVE $$$$
\question 
Prove that if $G$ is a group with the property that the square of every element is the identity, then $G$ is Abelian.

%%%% THIRTY SIX $$$$
\question 
Prove that the set of all $3\times3$ matrices with real entries of the form
\[
\begin{bmatrix}
    1 & a & b \\
    0 & 1 & c \\
    0 & 0 & 1
\end{bmatrix}
\]

is a group. (Multiplication is defined by 
\[
\begin{bmatrix}
    1 & a & b \\
    0 & 1 & c \\
    0 & 0 & 1
\end{bmatrix}
\begin{bmatrix}
    1 & a' & b' \\
    0 & 1 & c' \\
    0 & 0 & 1
\end{bmatrix}
=
\begin{bmatrix}
    1 & a+a' & b'+ac'+b  \\
    0 & 1    & c+c' \\
    0 & 0    & 1
\end{bmatrix}
\]

This group, sometimes called the \textit{Heisenberg group} after the Nobel Prize-winning physicist Werner Heisenber, 
is intimately related to the Heisenberg Uncertainty Principle of quantum physics.)

%%%% THIRTY SEVEN $$$$
\question 
Prove the assertion that the set $\{1,2,...,n-1\}$ is a group under multiplication modulo $n$ if and only if $n$ is 
prime.

%%%% THIRTY EIGHT $$$$
\question 
In a finite group, show that the number of nonidentity elements that satisfy the equation $x^5=e$ is a multiple of 4. 
If the stipluation that the group be finite is omitted, what can you say about the number of nonidentity elements 
that satisfy the equation $x^5=e$?

%%%% THIRTY NINE $$$$
\question 
Let $G=\left\{
\begin{bmatrix} 
    a & a \\ 
    a & a 
\end{bmatrix} | a \in \mathbb{R}, a \neq 0 \right\}$.
Show that $G$ is a group under matrix multiplication. Explain what each element of $G$ has inverse even though the matrices 
have 0 determinant. (Compare with Example 10.)



\end{document}








