\documentclass[11pt,largemargins]{homework}

\newcommand{\hwname}{------ ------}
\newcommand{\hwemail}{-}
\newcommand{\hwtype}{Chapter}
\newcommand{\hwnum}{0}
\newcommand{\hwclass}{Abstract Algebra}
\newcommand{\hwlecture}{0}
\newcommand{\hwsection}{Z}

\begin{document}
\maketitle

\textbf{\large{Important Statements:}}

\underline{Well Ordering Principle:}

\quad Every nonempty set of positive integers contains a smallest member

\underline{Division Algorithm:} 

\quad Let $a$ and $b$ be integers with $b>0$. Then there exist unique integers $q$ and $r$ with the 
property that $a=bq+r$, where $0\leq r<b$.

\underline{GCD is a Linear Combination:}

\quad For any nonzero integers $a$ and $b$, there exist integers $s$ and $t$ such that $\text{gcd}(a,b)=as+bt$.
Moreover, gcd$(a,b)$ is the smallest positive integer of the form $as+bt$.

\underline{Corollary:}

\quad If $a$ and $b$ are relatively prime, then there exist integers $s$ and $t$ such that $as+bt=1$.

\underline{Euclid's Lemma:}

\quad If $p$ is a prime that divides $ab$, then $p$ divides $a$ or $p$ divides $b$.

\underline{Fundemental Theorem of Arithmetic:}

\quad Every integer greater than 1 is a prime or a product of primes. This product is unique, except for the order 
in which the factors appear. That is, if $n=p_1p_2...p_r$ and $n=q_1q_2...q_s$ where the $p$'s and $q$'s are primes,
then $r=s$ and, after renumbering the $q$'s, we have $p_i = q_i$ for all $i$.

\underline{First Principle of Mathematical Induction:}

\quad Let $S$ be a set of integers containing $a$. Suppose $S$ has the property that whenever some integer 
$n\geq a$ belongs to $S$, then the integer $n+1$ also belongs to $S$. Then, $S$ contains every integer 
greater than or equal to $a$.

\underline{Second Principle of Mathematical Induction:}

\quad Let $S$ be a set of integers containing $a$. Suppose $S$ has the property that $n$ belongs to $S$ whenever 
every integer less than $n$ and greater than or equal to $a$ belongs to $S$. Then, $S$ contains every integer greater 
than or equal to $a$.

\underline{Equivalence Relations:}

\quad 1. $a \sim a \;\forall\; a \in S$ (Reflexive Property)

\quad 2. $a \sim b \Rightarrow b\sim a $ (Symmetric Property)

\quad 3. $a\sim b \land b\sim c \Rightarrow a\sim c $ (Transitive Property)

\underline{Equivalence Classes Partition:}

\quad The equivalence classes of an equivalence relation on a set $S$ constitute a partition of $S$. Conversely, 
for any partition $P$ of $S$, there is an equivalence relation on $S$ whose equivalence classes are the 
elements of $P$.

\underline{Properties of Functions:}

For $\alpha: \text{A}\rightarrow\text{B},\beta: \text{B}\rightarrow\text{C}\text{, and }\gamma: \text{C}\rightarrow\text{D}$

\quad 1. $\gamma(\beta\alpha) = (\gamma\beta)\alpha$.

\quad 2. $\alpha, \beta$ are one-to-one $\Rightarrow \beta\alpha$ is one-to-one.

\quad 3. $\alpha, \beta$ are onto $\Rightarrow \beta\alpha$ is onto.

\quad 4. $\alpha$ is one-to-one and onto $\Rightarrow \exists\; \alpha^{-1}:\text{B}\rightarrow\text{A}\;|\;
(\alpha^{-1}\alpha)(a)=a \;\forall\; a \in \text{A}$ and $(\alpha\alpha^{-1})(b)=b \;\forall\; b \in \text{B}$

\hfill

\textbf{\large{End of Chapter Exercises}}

%%%% ONE %%%%
\question
For $n=5,8,12,20,$ and $25,$ find all positive integers less than $n$ and relatively prime to $n$.

\begin{alphaparts}
    \questionpart $n=5$,
    $\{1,2,3,4\}$

    \questionpart $n=8$,
    $\{1,3,5,7\}$

    \questionpart $n=12$,
    $\{1,5,7,11\}$

    \questionpart $n=20$,
    $\{1,3,7,9,11,13,17,19\}$

    \questionpart $n=25$,
    $\{1,2,3,4,6,7,8,9,11,12,13,14,16,17,18,19,21,22,23,24\}$

\end{alphaparts}

%%%% TWO %%%%
\question 
Determine: 

\begin{alphaparts}
    \questionpart
    gcd$(2^4 \cdot 3^2 \cdot 7^2, 2 \cdot 3^3 \cdot 7 \cdot 11) = 2 \cdot 3^2 \cdot 7$

    \questionpart
    lcm$(2^3 \cdot 3^2 \cdot 5, 2 \cdot 3^3 \cdot 7 \cdot 11) = 2^3 \cdot 3^3 \cdot 5 \cdot 7 \cdot 11$

\end{alphaparts}

%%%% THREE %%%%
\question 
Determine:

\begin{alphaparts}
    \questionpart $51\text{ mod }13 = 3\cdot13+ 12\text{ mod }13=12$
    
    \questionpart $342\text{ mod }85 = 4\cdot85+ 2\text{ mod }85=2$
    
    \questionpart $62\text{ mod }15 = 4\cdot15+ 2\text{ mod }15=2$

    \questionpart $10\text{ mod }15 = 0\cdot15+ 10\text{ mod }15=10$

    \questionpart $82\cdot73\text{ mod }7$
    \begin{align*}
        82\cdot73\text{ mod }7 &= (11\cdot7+5)\cdot(10\cdot7+3)\text{ mod }7\\
                               &= 5 \cdot 3 \text{ mod }7\\
                               &= 15\text{ mod }7\\
                               &= 1
    \end{align*}

    \questionpart $51+68\text{ mod }7 $
    \begin{align*}
        51+68\text{ mod }7 &= (7\cdot7+2)+(9\cdot7+5)\text{ mod }7\\
                               &= 2 + 5 \text{ mod }7\\
                               &= 0
    \end{align*}

    \questionpart $35\cdot24\text{ mod }11 $
    \begin{align*}
        35\cdot24\text{ mod }11 &= (3\cdot11+2)\cdot(2\cdot11+2)\text{ mod }11\\
                               &= 3 \cdot 2 \text{ mod }11\\
                               &= 12 \text{ mod }11\\
                               &= 1
    \end{align*}

    \questionpart $47+68\text{ mod }11 $
    \begin{align*}
        47+68\text{ mod }11 &= (4\cdot11+3)+(6\cdot11+2)\text{ mod }11\\
                               &= 3 + 2 \text{ mod }11\\
                               &= 5
    \end{align*}

\end{alphaparts}

%%%% FOUR %%%%
\question 
Find integers $s$ and $t$ such that $1 = 7 \cdot s + 11 \cdot t$. Show that $s$ and $t$ are not unique.

\quad We see that, $1=7\cdot(-3)+11\cdot(2)$.

\quad But also that, $1=7\cdot(8)+11\cdot(-5)$.

\quad We can conject:
$$1=7\cdot(11n-3)+11\cdot(-7n+2), n\in\mathbb{Z}$$


%%%% FIVE %%%%
\question 
In Florida, the fourth and fifth digits from the end of a driver's license number give the year of birth. The last three 
digits for a male with birth month $m$ and a birth date $b$ are represented by $40(m-1)+b$. For females the digits are 
$40(m-1)+b+500$. Determine the dates of birth of people who have last five digits:

\begin{alphaparts}
    \questionpart 
    42218
    $$218 = 40(6-1)+18$$

    \questionpart 
    53953
    $$953 = 40(12-1)+13+500$$

\end{alphaparts}

%%%% SIX %%%%
\question 
For driver's license number issued in New York prior to September of 1992, the three digits preceding the last two of the 
number of a male with birth month $m$ and birth date $b$ are represented by $63m+2b$. For females the digits are 
$63m+2b+1$. Determine the dates of birth and sex(es) corresponding to the numbers:

\begin{alphaparts}
    \questionpart 
    248
    $$248 = 63(3)+2(29)+1$$

    \questionpart 
    601
    $$601 = 63(9)+2(17)$$

\end{alphaparts}

%%%% SEVEN %%%%
\question
Show that if $a$ and $b$ are positive integers, then $ab = \text{lcm} (a,b) \cdot \text{gcd} (a,b)$.

\quad By Fundemental Theorem of Arithmetic, 
$$a = {p_1}^{n_1}{p_2}^{n_2}...{p_k}^{n_k}$$
$$b = {p_1}^{m_1}{p_2}^{m_2}...{p_k}^{m_k}$$

\quad We have,
$$\text{gcd}(a,b) = {p_1}^{\alpha_1}{p_2}^{\alpha_2}...{p_k}^{\alpha_k}, \quad \alpha_i = \text{min}(n_i, m_i)$$

\quad And, 
$$\text{lcm}(a,b) = {p_1}^{\beta_1}{p_2}^{\beta_2}...{p_k}^{\beta_k}, \quad \beta_i = \text{max}(n_i, m_i)$$

\quad So,
\begin{align*}
    \text{gcd}(a,b)\cdot\text{lcm}(a,b) &= \prod_{i=1}^k {p_i}^{\alpha_i} \cdot \prod_{i=1}^k {p_i}^{\beta_i}\\
                                        &= \prod_{i=1}^k {p_i}^{\alpha_i}\cdot{p_i}^{\beta_i}\\
                                        &= \prod_{i=1}^k {p_i}^{\alpha_i + \beta_i}
\end{align*}

\quad Because we have $\alpha_i + \beta_i = m_i + n_i$
$$\text{gcd}(a,b)\cdot\text{lcm}(a,b) = \prod_{i=1}^k {p_i}^{\alpha_i + \beta_i} = \prod_{i=1}^k {p_i}^{m_i + n_i} = ab$$

%%%% EIGHT %%%%
\question 
Suppose $a$ and $b$ are integers that divide the integer $c$. If $a$ and $b$ are relatively prime, show that $ab$ divides $c$. 
Show, by example, that if $a$ and $b$ are not relatively prime, then $ab$ need not divide $c$. 

\quad Let $a|c$ and $b|c$, then there exists integers $s$ and $t$ such that $c = as$ and $c = bt$.

\quad We also have integers $f$ and $g$ such that $1= af + bg$

\quad We can write,
\begin{align*}
    c &= caf + cdg\\
      &= (bt)af + (as)dg\\
      &= ab(tf+sg)
\end{align*}

\quad Thus, $ab | c$

\quad Additionally, if $a=3$, $b=6$, $c=12$ then $a|c$ and $b|c$, but $ab\nmid c$

%%%% NINE %%%%
\question 
If $a$ and $b$ are integers and $n$ is a positive integer, prove that 
$$a \text{ mod } n = b \text{ mod } n \iff n | (a-b)$$

%%%% TEN %%%%
\question 
Let $a$ and $b$ be integers and $d = \text{gcd}(a,b)$. If $a=da'$ and $b=db'$, show that $\text{gcd}(a',b')=1$.

%%%% ELEVEN %%%%
\question 
Let $n$ be a fixed positive integer greater than 1. If $a\text{ mod }n=a'$ and $b\text{ mod }n=b'$, i.e. 
$$a\text{ mod }n=a' \implies a=a'+sn, \;s\in\mathbb{Z}$$
$$b\text{ mod }n=b' \implies b=b'+tn, \;t\in\mathbb{Z}$$

Prove that,
\begin{equation}\tag{0.11a}
    (a+b)\text{ mod }n = (a'+b')\text{ mod }n 
\end{equation}

\begin{align*}
    (a+b)\text{ mod }n &= (a'+sn) + (b'+tn) \text{ mod }n\\
                     &= (a'+b') + (s+t)n\text{ mod }n\\
                     &= (a'+b')\text{ mod }n 
\end{align*}

\begin{equation}\tag{0.11b}
    (a b)\text{ mod }n = (a' b')\text{ mod }n 
\end{equation}

\begin{align*}
    (a b)\text{ mod }n &= (a'+sn) \cdot (b'+tn) \text{ mod }n\\
                       &= (a'b') + (a't + b's)n + (st)n^2\text{ mod }n\\
                       &= (a'b')\text{ mod }n 
\end{align*}

%%%% TWELVE %%%%
\question 
Let $a$ and $b$ be positive integers and let $d = \text{gcd}(a,b)$ and $m = \text{lcm}(a,b)$. If $t$ divides both $a$ and $b$, 
prove that $t$ divides $d$. If $s$ is a multiple of both $a$ and $b$, prove that $s$ is a multiple of $m$. 

\quad First, $a=\alpha t$ and $b=\beta t$ for some $\alpha, \beta \in\mathbb{Z}$.

\quad Also, $d=\text{gcd}(a,b)=ax+by$ for some $x, y, \in\mathbb{Z}$.

\quad Thus,
\begin{align*}
    d &= ax+by\\
      &= (\alpha t)x+(\beta t)y\\
      &= t(\alpha x + \beta y)
\end{align*}

\quad So, $t$ divides $d$. $\square$

\hfill

\quad Next, $s = a'a$ and $s = b'b$, for some multiples $a', b' \in\mathbb{Z}$


%%%% THIRTEEN %%%%
\question 
Let $n$ and $a$ be positive integers and let $d=\text{gcd}(a,n)$. Show that  

\begin{equation}\tag{0.13}
    \exists \; x \; | \; ax\text{ mod }n=1 \iff d=1
\end{equation}

%%%% FOURTEEN %%%%
\question 
Show that $5n+3$ and $7n+4$ are relatively prime for all $n$.

\quad We demonstrate through induction:

\quad $n=1$:
$$5(1)+3=8\text{ and } 7(1)+4=11$$

\quad Obviously, for $n=1$, $5n+3$ and $7n+4$ are relatively prime.

\quad $n>1$:

\quad We assume the statement is true for $n$ and demonstrate that the statement is also true for $n+1$.

$$5(n+1)+3 = (5n+3)+5$$
$$7(n+1)+4 = (7n+4)+7$$

\quad Since $5n+3$ and $7n+4$ are relatively prime, we can write:
$$1 = s(5n+3)+t(7n+4)$$

%%%% FIFTEEN %%%%
\question 
Prove that every prime greater than 3 can be written in the form $6n+1$ or $6n+5$.

\quad We can take the contrapositive of the statement as:
$$p \notin \{6n+1, 6n+5\}, n\in\mathbb{N} \Rightarrow p \notin Primes$$

\quad It's easy to see that every natural number can be written as $6n+k$ with $n \in \mathbb{Z}$ and 
$k \in \{0,1,2,3,4,5\}$. If $k\in\{0,2,4\}$ then the resultant will be divisible by 2 and thus not prime.
Similarly, if $k=\{0,3\}$ then the resultant is divisible by 3.

\quad If, by assumption, we take $p$ prime, then it must be that $k\in\{1,5\}$.

%%%% SIXTEEN %%%%
\question 
Determine 

\begin{alphaparts}
    \questionpart 
    $7^{1000} = \text{ mod }6$

    \quad Notice, $7=1\text{ mod }6$. This makes our calculations very simple:
    \begin{align*}
        7^{1000} &= 1^{1000}\text{ mod }6\\
                 &= 1 \text{ mod }6\\
                 &= 1
    \end{align*}

    \questionpart 
    $6^{1001} = \text{ mod }7$
    \quad Notice again, $6=-1\text{ mod }7$. Thus,
    \begin{align*}
        6^{1001} &= -1^{1001}\text{ mod }7\\
                 &= -1(-1^2)^{500} \text{ mod }7\\
                 &= -1
    \end{align*}
\end{alphaparts}

%%%% SEVENTEEN %%%%
\question 
Let $a$, $b$, $s$, and $t$ be integers. If $a\text{ mod }st=b\text{ mod }st$, show that 
$a\text{ mod }s=b\text{ mod }s$, and $a\text{ mod }t=b\text{ mod }t$. What conditions on $s$ and $t$ 
is needed to make the converse true?

%%%% EIGHTEEN %%%%
\question 
Determine $8^{402}\text{ mod }5$.

\quad Notice first that $8^4 = 4096 = 1\text{ mod }5$. So,
\begin{align*}
    8^{402} &= 8^2(8^4)^{100}\\
            &= 64(1)^{100}\text{ mod }5\\
            &= 4 \text{ mod }5
\end{align*}


%%%% NINETEEN %%%%
\question 
Show that $\text{gcd}(a,bc)=1$ if and only if $\text{gcd}(a,b)=1$ and $\text{gcd}(a,c)=1$.

%%%% TWENTY %%%%
\question 
Let $p_1,p_2,...,p_n$ be primes. Show that $p_1p_2...p_n+1$ is divisible by none of these primes.

%%%% TWENTY ONE %%%%
\question 
Prove that there are infinitely many primes. \textit{(Hint: Use Question 20.)}

%%%% TWENTY TWO %%%%
\question 
For every positive integer $n$, prove that $1+2+...+n=n(n+1)/2$.

%%%% TWENTY THREE %%%%
\question 
For every positive integer $n$, prove that a set with exactly $n$ elements has exactly $2^n$ subsets (counting the empty 
set and the entire set)

%%%% TWENTY FOUR %%%%
\question 
For any positive integer $n$, prove that $2^n3^{2n}-1$ is always divisible by 17.

%%%% TWENTY FIVE %%%%
\question 
Prove that there is some positive integer $n$ such that $n, n+1, n+2,...,n+200$ are all composite.

%%%% TWENTY SIX %%%%
\question 
(Generalized Euclid's Lemma) If $p$ is a prime and $p$ divides $a_1a_2...a_n$, prove that $p$ divides $a_i$ for some $i$.

%%%% TWENTY SEVEN %%%%
\question 
Use the Generalized Euclid's Lemma to establish the uniqueness portion of the Fundemental Theorem of Arithmetic.

%%%% TWENTY EIGHT %%%%
\question What is the larget bet that cannot be made with chips worth \$7.00 and \$9.00? Verify that your answer is 
correct with both forms of induction.

%%%% TWENTY NINE %%%%
\question 
Prove that the First Principle of Mathematical Induction is a consequence of the Well Ordering Principle.

%%%% THIRTY %%%%
\question 
The Fibonacci numbers are $1,1,2,3,5,8,13,21,34,...$. In general, the Fibonacci numbers are defined by $f_1=1, f_2=1$, and for 
$n\geq3, f_n=f_{n-1}+f_{n-2}$. Prove that the $n$th Fibonacci number of $f_n$ satisfies $f_n<2^n$.

%%%% THIRTY ONE %%%%
\question 
In the cut "As" from \textit{Songs in the Key of Life}, Stevie Wonder mentions the equation $8\times8\times8\times8=4$. Find 
all integers $n$ for which this statement is true, modulo $n$.

%%%% THIRTY TWO %%%%
\question 
Prove that for every integer $n, n^3\text{ mod }6=n\text{ mod }6$.

\quad We use induction,

\quad $n=1$ is obvious as $1=1^3\text{ mod }6$

\quad $n>1$:

We assume that $n^3\text{ mod }6=n\text{ mod }6$

\begin{align*}
    (n+1)^3\text{ mod }6 &= n^3+3n^2+3n+1\text{ mod }6\\
                         &= n + 3n^2+3n+1\text{ mod }6\\
                         &= (n+1) + 3n(n+1)\text{ mod }6\\
                         &= (n+1)\text{ mod }6\\
\end{align*}
\quad Because, either $3n$ or $3(n+1)$ is a multiple of 6.


%%%% THIRTY THREE %%%%
\question 
If it were 2:00A.M. now, what time would it be 3735 hours from now?

\quad We say that 2:00A.M. is $2\text{ mod }24$ so the question reduces to 
$$2+3735\text{ mod }24$$
\begin{align*}
    2+3735\text{ mod }24 &= 3737\text{ mod }24\\
                         &= 155(24) + 17 \text{ mod }24\\
                         &= 17 \text{ mod }24
\end{align*}
\quad Thus the time would be 5:00P.M.

%%%% THIRTY FOUR %%%%
\question 
Determine the check digit for a money order with identification number 7234541780.

%%%% THIRTY FIVE %%%%
\question 
Suppose that in one of the noncheck positions of a money order number, the digit 0 is substituted for the digit 9 or vice versa.
Prove that this error will not be detected by the check digit. Prove that all other errors involving a single position 
are detected.

%%%% THIRTY SIX %%%%
\question 
Suppose that a money order identification number and check digit of \\21720421168 is erroneously copied as 27750421168. 
Will the check digit detect the error?

%%%% THIRTY SEVEN %%%%
\question 
A transposition error involving distinct adjacent digits is one of the form $...ab...\rightarrow...ba...$ with $a\neq b$. 
Prove that the money order check digit scheme will not detect such errors unless the check digit itself is transposed. 

%%%% THIRTY EIGHT %%%%
\question 
Determine the check digit for the Avis rental car with identification number 540047.

%%%% THIRTY NINE %%%%
\question 
Show that a substitution of a digit $a_i'$ for the digit $a_i(a_i'\neq a_i)$ in a noncheck position of a UPS number is 
detected if and only if $|a_i-a_i'|\neq 7$.

%%%% FORTY %%%%
\question 
Determine which transposition errors involving adjacent digits are detected by the UPS check digit.

%%%% FORTY ONE %%%%
\question 
Use the UPC scheme to determine the check digit for the number\\ 07312400508

%%%% FORTY TWO %%%%
\question 
Explain why the check digit for a money order for the number $N$ is the repeated decimal digit in the real 
number $N\div 9$.

%%%% FORTY THREE %%%%
\question 
The 10-digit International Standard Book Number (ISBN-10)\\ $a_1a_2a_3a_4a_5a_6a_7a_8a_9a_{10}$ has the property \\
$(a_1, a_2, ..., a_{10})\cdot (10,9,8,7,6,5,4,3,2,1)\text{ mod }11=0$. The digit $a_{10}$ is the check digit. 
When $a_{10}$ is required to be 1- to make the dot product 0, the character X is used as the check digit.
Verify the check digit for the ISBN-10 assigned to this book.

%%%% FORTY FOUR %%%%
\question 
Suppose that an ISBN=10 has a smudged entry where the question mark appears in the number 0-716?-2841-9.
Determine the missing digit.

%%%% FORTY FIVE %%%%
\question 
Suppose three consectutive digits $abc$ of an ISBN-10 are scrambled as $bca$. Which such errors will go undetected?

%%%% FORTY SIX %%%%
\question 
The ISBN-10 0-669-03925-4 is the result of a transposition of two adjacent digits not involving the first or last digit. 
Determine the correct ISBN-10.

%%%% FORTY SEVEN %%%%
\question 
Suppose the weighting vector for ISBN-10s was changed to (1,2,3,...,10). Explain how this would affect the check digit.

%%%% FORTY EIGHT %%%%
\question 
Use the two-check-digit error-correction method described in this chapter to append two check digits to the 
number 73445860.

%%%% FORTY NINE %%%%
\question 
Suppose that an eight-digit number has two check digits appended using the error-correction method described in this chapter 
and it is incorrectly transcribed as 4302511568. If exactly one digit is incorrect, determine the correct number.

%%%% FIFTY %%%%
\question 
The state of Utah appends a ninth digit $a_9$ to an eight-digit driver's license number $a_1a_2...a_8$ so that 
$(9a_1+8a_2+7a_3+6a_4+5a_5+4a_6+3a_7+2a_8+a_9)\text{ mod } 10 =0$. If you know that the license number 149105267 has 
exactly one digit incorrect, explain why the error cannot be in position 2,4,6, or 8.

%%%% FIFTY ONE %%%%
\question 
Complete the proof of Theorem 0.7

%%%% FIFTY TWO %%%%
\question 
Let $S$ be the set of real numbers. If $a,b \in S$, define $a\sim b$ if $a-b$ is an integer. Show that $\sim$ is an 
equivalence relation on $S$. Describe the equivalence classes of $S$.

%%%% FIFTY THREE %%%%
\question 
Let $S$ be the set of integers. If $a,b \in S$, define $aRb$ if $ab\geq 0$. Is $R$ an equivalence relation on $S$?

%%%% FIFTY FOUR %%%%
\question 
Let $S$ be the set of integers. If $a,b \in S$, define $aRb$ is $a+b$ is even. Prove that $R$ is an equivalence relation 
and determine the equivalence classes of $S$.

%%%% FIFTY FIVE %%%%
\question
Complete the proof of Theorem 0.6 by showing that $\sim$ is an equivalence relation on $S$.

%%%% FIFTY SIX %%%%
\question 
Prove that none of the integers 11, 111, 1111, 11111, ... is a square of an integer.

%%%% FIFTY SEVEN %%%%
\question 
(Cancellation Property) Suppose $\alpha, \beta, \text{ and, } \gamma$ are functions. If $\alpha\gamma=\beta\gamma$ and $\gamma$ 
is one-to-one and onto, prove that $\alpha=\beta$.







\end{document}








