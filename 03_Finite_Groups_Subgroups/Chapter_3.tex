\documentclass[11pt,largemargins]{homework}

\newcommand{\hwname}{------ ------}
\newcommand{\hwemail}{-}
\newcommand{\hwtype}{Chapter}
\newcommand{\hwnum}{3}
\newcommand{\hwclass}{Abstract Algebra}
\newcommand{\hwlecture}{0}
\newcommand{\hwsection}{Z}

\begin{document}
\maketitle

\textbf{\large{Important Statements:}}

\underline{One-Step Subgroup Test:}

\quad Let $G$ be a group and $H$ a nonempty subset of $G$. If $ab^{-1}$ is in $H$ whenever $a$ and $b$ are in $H$,
then $H$ is a subgroup of $G$. (In additive notation, if $a-b$ is in $H$ whenever $a$ and $b$ are in $H$, then $H$ is 
a subgroup of $G$.)

\underline{Two-Step Subgroup Test:}

\quad Let $G$ be a group and let $H$ be a nonempty subset of $G$. If $ab$ is in $H$ whenever $a$ and $b$ are in $H$
($H$ is closed under the operation), and $a^{-1}$ is in $H$ whenever $a$ is in $H$ ($H$ is closed under taking inverses),
then $H$ is a subgroup of $G$.


\underline{Finite Subgroup Test:}

\quad Let $H$ be a nonempty finite subset of a group $G$. If $H$ is closed under the operation of $G$, then 
$H$ is a subgroup of $G$.

\underline{$\langle a \rangle$ Is a Subgroup:}

\quad Let $G$ be a group, and let $a$ be any element of $G$. Then, $\langle a \rangle$ is a subgroup of $G$.

\underline{Center of a Group:}

\quad The \textit{center}, $Z(G)$, of a group $G$ is the subset of elements in $G$ that commute with every 
element of $G$. i.e.
$$Z(G) = \{a \in G \;|\; ax=xa \text{ for all } x \text{ in } G\}$$

\underline{Center is a Subgroup:}

\quad The center of a group $G$ is a subgroup of $G$.

\underline{Centralizer of $a$ in $G$:}

\quad Let $a$ be a fixed element of a group $G$. The \textit{centralizer} of $a$ in $G$, $C(a)$, is the set of 
all elements in $G$ that commute with $a$. i.e.
$$C(a) = \{ g \in G \;|\; ga=ag \}$$

\underline{$C(a)$ is a Subgroup:}

\quad For each $a$ in a group $G$, the centralizer of $a$ is a subgroup of $G$.

\underline{Notation:}

\quad $\langle a \rangle = \{a^n \;|\; n\in \mathbb{Z}\}$

\hfill

\textbf{\large{End of Chapter Exercises}}

%%%% ONE %%%%
\question
For each group in the following list, find the order of the group and the order of each element in the group.
What relation do you see between the orders of the elements of a group and the order of the group?

\begin{alphaparts}
    \questionpart
    $\mathbb{Z}_{12} = \{0,1,2,3,4,5,6,7,8,9,10,11\}$ 

    \questionpart
    $U(10) = \{1,3,7,9\}$

    \questionpart
    $U(12) = \{1,5,7,11\}$

    \questionpart
    $U(20) = \{1,3,7,9,11,13,17,19\}$

    \questionpart
    $D_4 = \{R_0,R_{90},R_{180},R_{270},H, V, D, D'\}$

\end{alphaparts}

\quad The order of each element of a group is a divisor of the order of the group.

%%%% TWO %%%%
\question
Let $\mathbb{Q}$ be the group of rational numbers under addition and let $\mathbb{Q}^{\star}$ be the group of nonzero rational numbers under 
multiplication. In $\mathbb{Q}$, list the elements in $\langle \frac{1}{2} \rangle$. In $\mathbb{Q}^{\star}$, list the elements 
in $\langle \frac{1}{2} \rangle$.

\quad $\langle \frac{1}{2} \rangle$ in $\mathbb{Q} = \{ \frac{n}{2} \;|\; n\in\mathbb{Z}\}$

\quad $\langle \frac{1}{2} \rangle$ in $\mathbb{Q}^{\star} = \{ \left( \frac{1}{2}\right)^n  \;|\; n\in\mathbb{Z}\}$

%%%% THREE %%%%
\question
Let $\mathbb{Q}$ and $\mathbb{Q}^{\star}$ be as in Question 2. Find the order of each element in $\mathbb{Q}$ 
and in $\mathbb{Q}^{\star}$.

\quad The elements of $\mathbb{Q}$ and $\mathbb{Q}^{\star}$ have infinite order, except their respective identity elements.

%%%% FOUR %%%%
\question
Prove that in any group, an element and its inverse have the same order.

\quad Let $G$ be a group with element $a$, and $|a| = n$. Then, $n$ is the lowest integer for which $a^n=e$.

\begin{align*}
    e &= e^{-1}\\
      &= (a^n)^{-1}\\
      &= (a*a*\dots*a)^{-1}\\
      &= a^{-1}*a^{-1}*\dots*a^{-1}\\
      &= (a^{-1})^n
\end{align*}

%%%% FIVE %%%%
\question
Without actually computing the orders, explain why the two elements in each of the following pairs of elements from $\mathbb{Z}_{30}$ 
must have the same order: $\{2,28\}, \{8,22\}$. Do the same for the following pairs of elements from $U(15):\{2,8\}, \{7,13\}$.

\quad For $\{2,28\}, \{8,22\}$, it is easy to see that each element is the inverse of its corresponding element. By the previous 
Question, they have the same order.

\quad For $U(15):\{2,8\}, \{7,13\}$, a similar argument holds. 

\quad More thoroughly, $2+28 = 0\text{ mod }30$ and $8+22 = 0\text{ mod }30$. Also $2*8=16=1\text{ mod }15$ and 
$7*13=91=1\text{ mod }15$.

%%%% SIX %%%%
\question
Suppose that $a$ is a group element and $a^6=e$. What are the possibilities for $|a|$? Provide reasons for your answer.

\quad The posibilities for $|a|$ are 1,2,3, and 6. 

If $a$ itself is the identity element, then $|a|=1$. 

If $a^2=e$, then $a^6=(a^2)^3=e^3=e$. And similarly for $|a|=3$.

If $|a|=4$, then $a^6=e*a^2\neq e$. Also, if $|a|=5$, then $a^6=e*a\neq e$. So neither 4 or 5 are the order of $a$.

%%%% SEVEN %%%%
\question
If $a$ is a group element and $a$ has infinite order, prove that $a^m\neq a^n$ when $m\neq n$.

\quad Without loss of generality, assume $m>n$. For the purpose of contradiction, assume $a^m=a^n$.
\begin{align*}
    a^m &= a^n*a^{m-n}               &    &\text{Associativity}\\
        &= a^m*a^{m-n}               &    &\text{Equality of $m$ and $n$}\\
        &= a^n*a^{m-n}*a^{m-n}       &    &\text{Associativity}\\
        &= a^m*a^{m-n}*\dots*a^{m-n} &    &\text{Repetition of above}
\end{align*}

\quad Because we know that $a^{m-n}\neq e$ (else either $|a|=m-n$, or $m=n$ contradicting our assumption), and $a^{m-n}=e$ is the 
only way the above equalities hold, our presupposition must be false and so $a^m\neq a^n$. $\square$


%%%% EIGHT %%%%
\question
Let $x$ belong to a group. If $x^2\neq e$ and $x^6=e$, prove that $x^4\neq e$ and $x^5\neq e$. What can we say about the order of $x$?

%%%% NINE %%%%
\question
Show that if $a$ is an element of a group $G$, then $|a|\leq |G|$.

%%%% TEN %%%%
\question
Show that $U(14)=\langle 3 \rangle = \langle 5 \rangle$. (Hence, $U(14)$ is cyclic.) Is $U(14)=\langle 11 \rangle$?

%%%% ELEVEN %%%%
\question
Show that $U(20)\neq \langle k \rangle$ for any $k$ in $U(20)$. (Hence, $U(20)$ is not cyclic.)

%%%% TWELVE %%%%
\question
Prove that an Abelian group with two elements of order 2 must have a subgroup of order 4.

%%%% THIRTEEN %%%%
\question
Find groups that contain elements $a$ and $b$ such that $|a|, |b|,$ and $|ab|$?
\begin{alphaparts}
    \questionpart
    $|ab|=3$

    \questionpart
    $|ab|=4$

    \questionpart
    $|ab|=5$

\end{alphaparts}

%%%% FOURTEEN %%%%
\question
Suppose that $H$ is a proper subgroup of $\mathbb{Z}$ under addition and that $H$ contains 18, 30, and 40. 
What are the possibilities for $H$?

%%%% FIFTEEN %%%%
\question
Suppose that $H$ is a proper subgroup of $\mathbb{Z}$ under addition and that $H$ contains 12, 30, and 54. 
What are the possibilities for $H$?

%%%% SIXTEEN %%%%
\question
Prove that the dihedral group of order 6 does not have a subgroup of order 4.

%%%% SEVENTEEN %%%%
\question
For each divisor $k>1$ of $n$, let $U_k(n)=\{x\in U(n) \;|\; x \text{ mod } k = 1\}$. 
(For example, $U_3(21)=\{1,4,10,13,16,19\}$ and $U_7(21)=\{1,8\}$.)
List the elements of 
\begin{alphaparts}
    \questionpart
    $U_4(20)$
    \questionpart
    $U_5(20)$
    \questionpart
    $U_5(30)$
    \questionpart
    $U_{10}(30)$

\end{alphaparts}

Prove that $U_k(n)$ is a subgroup of $U(n)$. 

Let $H=\{x\in U(10)\;|\; x\text{ mod }3=1\}$. Is $H$ a subgroup of $U(10)$? 

%%%% EIGHTEEN %%%%
\question
If $H$ and $K$ are subgroups of $G$, show that $H\cap K$ is a subgroup of $G$. (Can you see that the same proof shows that the 
intersection of any number of subgroups of $G$, finite or infinite, is again a subgroup of $G$?)

%%%% NINETEEN %%%%
\question
Let $G$ be a group. Show that $Z(G)= \cap_{a\in G}C(a)$. (This means that intersection of all subgroups 
of the gorm $C(a)$.)

%%%% TWENTY %%%%
\question
Let $G$ be a group, and let $a\in G$. Prove that $C(a)=C(a^{-1})$.

%%%% TWENTY ONE %%%%
\question
For any group element $a$ and any integer $k$, show that $C(a)\subseteq C(a^k)$.
Use this fact to complete the following statement: "In a group, if $R$ is an integer and $x$ commutes with $a$, then ...?"
Is the converse true?

%%%% TWENTY TWO %%%%
\question
Complete the partial Cayley group table given below.

%%%% TWENTY THREE %%%%
\question
Suppose $G$ is the group defined by the following Cayley table.

\begin{alphaparts}
    \questionpart
    Find the centralizer of each memeber of $G$.
    \questionpart
    Find $Z(G)$.
    \questionpart
    Find the order of each element of $G$. How are these orders arithmetically related to the order of the group?
\end{alphaparts}

%%%% TWENTY FOUR %%%%
\question
If $a$ and $b$ are distinct group elements, prove that either $a^2\neq b^2$ or $a^3\neq b^3$.

%%%% TWENTY FIVE %%%%
\question
Prove Theorem 3.6

%%%% TWENTY SIX %%%%
\question
Prove that $C(H)$ is a subgroup of $G$.

%%%% TWENTY SEVEN %%%%
\question
Must the centralizer of an element of a group be Abelian?

%%%% TWENTY EIGHT %%%%
\question
Must the center of a group be Abelian?

%%%% TWENTY NINE %%%%
\question
Let $G$ be an Abelian group with identity $e$ and let $n$ be some fixed integer. Prove that the set of all elements 
of $G$ that satisfy the equation $x^n=e$ is a subgroup of $G$. Give an example of a group $G$ in which the set of all elements of 
$G$ that satisfy the equation $x^2=e$ does not form a subgroup of $G$.

%%%% THIRTY %%%%
\question
Suppose $a$ belongs to a group and $|a|=5$. Prove that $C(a)=C(a^3)$. Find an element $a$ from some group such 
that $|a|=6$ and $C(a)\neq C(a^3)$.

%%%% THIRTY ONE %%%%
\question
Determine all finite subgroups of $\mathbb{R}^{\star}$, the group of nonzero real numbers under multiplication.

%%%% THIRTY TWO %%%%
\question
Suppose $n$ is an even positive integer and $H$ is a subgroup of $\mathbb{Z}_n$. Prove that either every member of $H$ is 
even or exactly half of the members of $H$ are even.

%%%% THIRTY THREE %%%%
\question
Suppose a group contains elements $a$ and $b$ such that $|a|=4$, $|b|=2$, and $a^3b=ba$. Find $|ab|$.

%%%% THIRTY FOUR %%%%
\question
Suppose $a$ and $b$ are group elements such that $|a|=2, b\neq e,$ and $aba=b^2$. Determine $|b|$.

%%%% THIRTY FIVE %%%%
\question
Let $a$ be a group element of order $n$, and suppose that $d$ is a positive divisor of $n$. Prove that $|a^d|=n/d$.

%%%% THIRTY SIX %%%%
\question

%%%% THIRTY SEVEN %%%%
\question

%%%% THIRTY EIGHT %%%%
\question

%%%% THIRTY NINE %%%%
\question
Let $G$ be the symmetry group of a circle. Show that $G$ has elements of every finite order as well as 
elements of infinite order.

%%%% FORTY %%%%
\question
Let $x$ belong to a group of $|x|=6$. Find $|x^2|,|x^3|,|x^4|,$ and $|x^5|$. Let $y$ belong to a group 
and $|y|=9$. Find $|y^i|$ for $i=2,3,...,8$. Do these examples suggest any relationship between the order of the power 
of an element and the order of the element?

%%%% FORTY ONE %%%%
\question

%%%% FORTY TWO %%%%
\question

%%%% FORTY THREE %%%%
\question

%%%% FORTY FOUR %%%%
\question

%%%% FORTY FIVE %%%%
\question

%%%% FORTY SIX %%%%
\question

%%%% FORTY SEVEN %%%%
\question

%%%% FORTY EIGHT %%%%
\question

%%%% FORTY NINE %%%%
\question

%%%% FIFTY %%%%
\question

%%%% FIFTY ONE %%%%
\question

%%%% FIFTY TWO %%%%
\question

%%%% FIFTY THREE %%%%
\question

%%%% FIFTY FOUR %%%%
\question

%%%% FIFTY FIVE %%%%
\question

%%%% FIFTY SIX %%%%
\question

%%%% FIFTY SEVEN %%%%
\question

%%%% FIFTY EIGHT %%%%
\question

%%%% FIFTY NINE %%%%
\question

%%%% SIXTY %%%%
\question

%%%% SIXTY ONE %%%%
\question

%%%% SIXTY TWO %%%%
\question







\end{document}








